% !TEX program = pdflatex
\documentclass[11pt,a4paper]{article}

% Encoding and language
\usepackage[utf8]{inputenc}
\usepackage[T1]{fontenc}
\usepackage[english]{babel}

% Math
\usepackage{amsmath,amssymb,amsthm}

% Page layout
\usepackage{geometry}
\geometry{a4paper,margin=2.5cm}

% Graphics and hyperlinks
\usepackage{graphicx}
\usepackage{hyperref}
\hypersetup{
	colorlinks=true,
	linkcolor=blue,
	citecolor=blue,
	urlcolor=blue
}

% Theorem environments (customize as needed)
\theoremstyle{plain}
\newtheorem{theorem}{Theorem}[section]
\newtheorem{lemma}[theorem]{Lemma}
\newtheorem{proposition}[theorem]{Proposition}
\newtheorem{corollary}[theorem]{Corollary}

\theoremstyle{definition}
\newtheorem{definition}[theorem]{Definition}
\newtheorem{example}[theorem]{Example}

\theoremstyle{remark}
\newtheorem{remark}[theorem]{Remark}

% Title information
\title{MAT740 Topology Report Instructions}
\date{\today}

\begin{document}

\maketitle

\section{Aims of the report}
Your report should document a small extension of the topology library developed in the course MAT740. Your goal is to formalize a concrete piece of topology and to prove some representative results about it, working with the course definitions rather than the general topological library of \texttt{Mathlib}.

The main goal is to explain how your results are represented and verified in the theorem prover Lean. Your report should be written for a mathematician who is not familiar with Lean: you should spend some space introducing the underlying mathematics and its significance, but devote most of the discussion to the choices you made in your formalization, as well as the explaining the basics of theorem proving in lean.

\section{Report structure}
Your report should be at most five pages. Use a clear structure, using sections and subsections. One possible outline is the following:

\begin{enumerate}
    \item \textbf{Abstract.} Summarize the results of your report in a single paragraph. Use the \texttt{abstract} environment.

    \item \textbf{Introduction.} Introduce the mathematics you are formalizing and explain why it is interesting or important. State the aims of your formalization and briefly sketch the main steps you took in Lean. Write this after completing the rest of your report.

    \item \textbf{Background on Lean.} Give a short introduction to theorem proving in Lean that is sufficient for a mathematician unfamiliar with Lean to roughly follow your work.
    In particular, you should
    \begin{itemize}
        \item explain the basic workflow of writing and checking proofs in Lean;
        \item give a high-level description of how tactics are used (you do not need to explain every tactic you use);
        \item if you wish, include some theoretical justification in terms of dependent type theory (see, for example, the online text \emph{Theorem Proving in Lean~4}).
    \end{itemize}

    \item \textbf{Results.} Describe the Lean code you wrote and how it is organized. Focus on the most important and representative parts rather than every detail.
    You should
    \begin{itemize}
        \item explain the main definitions you introduced and the statements of the key theorems you proved;
        \item indicate where in your files these definitions and theorems can be found (using filenames and line numbers);
        \item include short code snippets to illustrate the style of your formalization, but you should not reproduce all of your code in the report.
    \end{itemize}

    \item \textbf{Discussion.} Reflect on the formalization process.
    For example, you can
    \begin{itemize}
        \item discuss any nuances or difficulties that arose and how you addressed them;
        \item comment on design choices (both mathematical and in the Lean code);
        \item explain what you learned from the project and what you might do differently next time.
    \end{itemize}

    \item \textbf{References.} Include references to sources for the mathematics (e.g.~textbooks or lecture notes containing full proofs) and to Lean/mathlib documentation. You should also cite Lean~4 itself (see the instructions under ``How to cite Lean'' \href{https://lean-lang.org/learn/}{here}).
    
    \item \textbf{AI usage declaration.} If you used AI to help you write your code or report, declare this for the sake of transparency (what model / how did you use it). If you want, you can try using \href{https://ai-cards.org}{ai cards}. The extent of your AI use will \emph{not} affect your grade.
\end{enumerate}

\end{document}

